\section{Methodology}

A qualitative methodology was adopted to conduct this study. In particular, different legal and technical documentation and publications were taken into consideration.

To search the technical material concerning the computer science domain, the following (non-exhaustive) list of keywords was used on Google Scholar and ReasearchGate: classifier for adult children, ML for CSAM detection, PhotoDNA, client-side scanning, and AI classifier for grooming.

Concerning the research related to the juridical aspects, the following keywords were used on EUR-Lex: CSAM, CSAM Regulation, and privacy. Moreover, concerning the search for judiciary cases, the following keywords were used on curia.europa.eu: personal data processing, fundamental rights, ePrivacy, privacy.

Furthermore, to gather additional data on judiciary cases relating to privacy, thematic factsheets for the press released by the ECHR\footnote{see \url{https://prd-echr.coe.int/web/echr/factsheets}} were taken into consideration as indexes for relevant rulings. In particular, factsheets "Mass surveillance" and "Protection of personal data" were considered.

The reason for the review of these documents was to assess the technical efficiency, efficacy, and feasibility of the solution in terms of technology. Moreover, the analysis relating to the juridical aspect was focused partly on the proportionality of the proposal and the interpretation of Article 8 of the Charter of Fundamental Rights of the European Union about both the examined judiciary cases and the CSAM proposal.
