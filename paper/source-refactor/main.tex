%%%%%%%% Sample LaTeX input for Complex Systems %%%%%%%%%%% 
% Revision 4, Jun 27, 2018
%
% This is a LaTeX input file  
% Text following % on a particular line is treated as a comment, and 
% ignored by LaTeX.  
% You do not need to type any text that follows a % 
% 
\documentclass[10pt]{article}

\usepackage[a4paper, margin=1.5in]{geometry}

\usepackage{epsf,hyperref}
\usepackage{amssymb,ComplexSystems}

% Font Encoding and Input
\usepackage[T1]{fontenc}  % Updated to use T1 encoding for better font rendering
\usepackage[utf8]{inputenc}  % Use utf8 instead of utf8x for better compatibility
\usepackage{lmodern}  % Improved font for better scalability

% Language
\usepackage[english]{babel}

% Code Listings
\usepackage{listings}
\usepackage{color}
\usepackage[table,xcdraw]{xcolor}

% Graphics and Figures
\usepackage{graphicx}
\usepackage{wrapfig}
\usepackage{subcaption}
\usepackage{caption}

% Clever References
\usepackage{cleveref}

% Bibliography
\usepackage{biblatex}
\addbibresource{biblio.bib}

% complex-systems.sty is the macro package for Complex Systems.
% It is available at
% http://www.complex-systems.com/samples/complex-systems.sty
% epsf.sty is the preferred graphics import method

\begin{document}

\title{An Overview on the Chatcontrol Proposal% 
% Use \\ to indicate line breaks in titles longer than about 
% 55 characters. 
%
}

\author{\authname{Riccardo Gennaro}\\[2pt] 
% Use \\[2pt] to end the line and add space between author name and affiliation. 
\authadd{Department of Information Engineering and Computer Science, University of Trento}\\
\authadd{Via Sommarive, 9}\\
\authadd{Trento, TN 38123, Italy}\\
}

% The following specifies the running headings 
%
% Each running heading should be less than about 50 characters long. 
% If necessary, give a shortened version of the title. 
%
% Use initials for first and second names. If all author names do not fit, truncate the 
% list and end with ``et al.''.
\markboth{Chatcontrol Proposal} 
{An Overview on the Chatcontrol Proposal} 

\maketitle
% End title section

\begin{abstract}

\end{abstract}

\begin{keywords}

\end{keywords}

\section*{Abstract}
\section*{Abstract}
\section*{Abstract}
\section*{Abstract}
\section*{Abstract}
\section*{Abstract}

\printbibliography

\end{document}
