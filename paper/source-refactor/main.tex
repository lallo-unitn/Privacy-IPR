%%%%%%%% Sample LaTeX input for Complex Systems %%%%%%%%%%% 
% Revision 4, Jun 27, 2018
%
% This is a LaTeX input file  
% Text following % on a particular line is treated as a comment, and 
% ignored by LaTeX.  
% You do not need to type any text that follows a % 
% 
\documentclass[10pt]{article}

\usepackage[a4paper, margin=1.5in]{geometry}

\usepackage{epsf,hyperref}
\usepackage{amssymb,ComplexSystems}

% Font Encoding and Input
\usepackage[T1]{fontenc}  % Updated to use T1 encoding for better font rendering
\usepackage[utf8]{inputenc}  % Use utf8 instead of utf8x for better compatibility
\usepackage{lmodern}  % Improved font for better scalability

% Language
\usepackage[english]{babel}

% Code Listings
\usepackage{listings}
\usepackage{color}
\usepackage[table,xcdraw]{xcolor}

% Graphics and Figures
\usepackage{graphicx}
\usepackage{wrapfig}
\usepackage{subcaption}
\usepackage{caption}

% Clever References
\usepackage{cleveref}

% Bibliography
\usepackage{biblatex}
\addbibresource{biblio.bib}

% complex-systems.sty is the macro package for Complex Systems.
% It is available at
% http://www.complex-systems.com/samples/complex-systems.sty
% epsf.sty is the preferred graphics import method

\begin{document}

\title{An Overview on the Chatcontrol Proposal% 
% Use \\ to indicate line breaks in titles longer than about 
% 55 characters. 
%
}

\author{\authname{Riccardo Gennaro}\\[2pt] 
% Use \\[2pt] to end the line and add space between author name and affiliation. 
\authadd{Department of Information Engineering and Computer Science, University of Trento}\\
\authadd{Via Sommarive, 9}\\
\authadd{Trento, TN 38123, Italy}\\
}

% The following specifies the running headings 
%
% Each running heading should be less than about 50 characters long. 
% If necessary, give a shortened version of the title. 
%
% Use initials for first and second names. If all author names do not fit, truncate the 
% list and end with ``et al.''.
\markboth{Chatcontrol Proposal} 
{An Overview on the Chatcontrol Proposal} 

\maketitle
% End title section

\begin{abstract}

\end{abstract}

\begin{keywords}

\end{keywords}

\section{Conclusion}
\label{s:concl}

This work offered an overview of both the legislative and technical problems related to the European Commission CSAM Regulation Proposal, also known as Chatcontrol 2.0.

In particular, to understand the degree of interference that this proposal would cause to the privacy rights of EU citizens, both current related EU regulations and directives were taken into consideration. Furthermore, to interpret some fundamental articles, recent judiciary cases were reviewed.

Moreover, to assess the feasibility, efficacy, and efficiency of the proposed infrastructure from a technical standpoint, research data on classifiers, hashing algorithms, and content-scanning technologies were gathered. In particular, these data were critically compared with the ones reported in the impact assessment of the Commission and assessment of the Parliament.

While this work has limitations in terms of data analyzed and the expertise of the author in terms of legislative and judicial matters, it is my conclusion that this proposition presents an unbalanced solution to the problem of CSAM online spreading and sexual solicitation of children.

This is, not only because the control could easily be bypassed by those who had sufficient reason to learn how to do it, e.g. criminals, but also because the interference with the Fundamental Rights of the Union would not be proportional with the aim, the efficacy, and the efficiency of this proposal.

Ultimately, while researching this matter, not only did it become obvious how some consensus has to be reached on the balance between child safety and privacy, but also it became apparent that more research has to be carried out on non-intrusive (if any exists) or less intrusive technology for crime detection in digital communications.
\section{Conclusion}
\label{s:concl}

This work offered an overview of both the legislative and technical problems related to the European Commission CSAM Regulation Proposal, also known as Chatcontrol 2.0.

In particular, to understand the degree of interference that this proposal would cause to the privacy rights of EU citizens, both current related EU regulations and directives were taken into consideration. Furthermore, to interpret some fundamental articles, recent judiciary cases were reviewed.

Moreover, to assess the feasibility, efficacy, and efficiency of the proposed infrastructure from a technical standpoint, research data on classifiers, hashing algorithms, and content-scanning technologies were gathered. In particular, these data were critically compared with the ones reported in the impact assessment of the Commission and assessment of the Parliament.

While this work has limitations in terms of data analyzed and the expertise of the author in terms of legislative and judicial matters, it is my conclusion that this proposition presents an unbalanced solution to the problem of CSAM online spreading and sexual solicitation of children.

This is, not only because the control could easily be bypassed by those who had sufficient reason to learn how to do it, e.g. criminals, but also because the interference with the Fundamental Rights of the Union would not be proportional with the aim, the efficacy, and the efficiency of this proposal.

Ultimately, while researching this matter, not only did it become obvious how some consensus has to be reached on the balance between child safety and privacy, but also it became apparent that more research has to be carried out on non-intrusive (if any exists) or less intrusive technology for crime detection in digital communications.
\section{Conclusion}
\label{s:concl}

This work offered an overview of both the legislative and technical problems related to the European Commission CSAM Regulation Proposal, also known as Chatcontrol 2.0.

In particular, to understand the degree of interference that this proposal would cause to the privacy rights of EU citizens, both current related EU regulations and directives were taken into consideration. Furthermore, to interpret some fundamental articles, recent judiciary cases were reviewed.

Moreover, to assess the feasibility, efficacy, and efficiency of the proposed infrastructure from a technical standpoint, research data on classifiers, hashing algorithms, and content-scanning technologies were gathered. In particular, these data were critically compared with the ones reported in the impact assessment of the Commission and assessment of the Parliament.

While this work has limitations in terms of data analyzed and the expertise of the author in terms of legislative and judicial matters, it is my conclusion that this proposition presents an unbalanced solution to the problem of CSAM online spreading and sexual solicitation of children.

This is, not only because the control could easily be bypassed by those who had sufficient reason to learn how to do it, e.g. criminals, but also because the interference with the Fundamental Rights of the Union would not be proportional with the aim, the efficacy, and the efficiency of this proposal.

Ultimately, while researching this matter, not only did it become obvious how some consensus has to be reached on the balance between child safety and privacy, but also it became apparent that more research has to be carried out on non-intrusive (if any exists) or less intrusive technology for crime detection in digital communications.
\section{Conclusion}
\label{s:concl}

This work offered an overview of both the legislative and technical problems related to the European Commission CSAM Regulation Proposal, also known as Chatcontrol 2.0.

In particular, to understand the degree of interference that this proposal would cause to the privacy rights of EU citizens, both current related EU regulations and directives were taken into consideration. Furthermore, to interpret some fundamental articles, recent judiciary cases were reviewed.

Moreover, to assess the feasibility, efficacy, and efficiency of the proposed infrastructure from a technical standpoint, research data on classifiers, hashing algorithms, and content-scanning technologies were gathered. In particular, these data were critically compared with the ones reported in the impact assessment of the Commission and assessment of the Parliament.

While this work has limitations in terms of data analyzed and the expertise of the author in terms of legislative and judicial matters, it is my conclusion that this proposition presents an unbalanced solution to the problem of CSAM online spreading and sexual solicitation of children.

This is, not only because the control could easily be bypassed by those who had sufficient reason to learn how to do it, e.g. criminals, but also because the interference with the Fundamental Rights of the Union would not be proportional with the aim, the efficacy, and the efficiency of this proposal.

Ultimately, while researching this matter, not only did it become obvious how some consensus has to be reached on the balance between child safety and privacy, but also it became apparent that more research has to be carried out on non-intrusive (if any exists) or less intrusive technology for crime detection in digital communications.
\section{Conclusion}
\label{s:concl}

This work offered an overview of both the legislative and technical problems related to the European Commission CSAM Regulation Proposal, also known as Chatcontrol 2.0.

In particular, to understand the degree of interference that this proposal would cause to the privacy rights of EU citizens, both current related EU regulations and directives were taken into consideration. Furthermore, to interpret some fundamental articles, recent judiciary cases were reviewed.

Moreover, to assess the feasibility, efficacy, and efficiency of the proposed infrastructure from a technical standpoint, research data on classifiers, hashing algorithms, and content-scanning technologies were gathered. In particular, these data were critically compared with the ones reported in the impact assessment of the Commission and assessment of the Parliament.

While this work has limitations in terms of data analyzed and the expertise of the author in terms of legislative and judicial matters, it is my conclusion that this proposition presents an unbalanced solution to the problem of CSAM online spreading and sexual solicitation of children.

This is, not only because the control could easily be bypassed by those who had sufficient reason to learn how to do it, e.g. criminals, but also because the interference with the Fundamental Rights of the Union would not be proportional with the aim, the efficacy, and the efficiency of this proposal.

Ultimately, while researching this matter, not only did it become obvious how some consensus has to be reached on the balance between child safety and privacy, but also it became apparent that more research has to be carried out on non-intrusive (if any exists) or less intrusive technology for crime detection in digital communications.
\section{Conclusion}
\label{s:concl}

This work offered an overview of both the legislative and technical problems related to the European Commission CSAM Regulation Proposal, also known as Chatcontrol 2.0.

In particular, to understand the degree of interference that this proposal would cause to the privacy rights of EU citizens, both current related EU regulations and directives were taken into consideration. Furthermore, to interpret some fundamental articles, recent judiciary cases were reviewed.

Moreover, to assess the feasibility, efficacy, and efficiency of the proposed infrastructure from a technical standpoint, research data on classifiers, hashing algorithms, and content-scanning technologies were gathered. In particular, these data were critically compared with the ones reported in the impact assessment of the Commission and assessment of the Parliament.

While this work has limitations in terms of data analyzed and the expertise of the author in terms of legislative and judicial matters, it is my conclusion that this proposition presents an unbalanced solution to the problem of CSAM online spreading and sexual solicitation of children.

This is, not only because the control could easily be bypassed by those who had sufficient reason to learn how to do it, e.g. criminals, but also because the interference with the Fundamental Rights of the Union would not be proportional with the aim, the efficacy, and the efficiency of this proposal.

Ultimately, while researching this matter, not only did it become obvious how some consensus has to be reached on the balance between child safety and privacy, but also it became apparent that more research has to be carried out on non-intrusive (if any exists) or less intrusive technology for crime detection in digital communications.

\printbibliography

\end{document}
