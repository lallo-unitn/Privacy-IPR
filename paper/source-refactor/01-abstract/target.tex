This work explores the recent legislative proposal aimed at combating the proliferation of \textit{Child Sexual Abuse Material (CSAM)} and online child sexual solicitation, also referred to as grooming. The proposal adopted by the European Commission suggests employing bulk interception of communication and the use of machine learning classifiers and hashing algorithms to detect and report such criminal behaviors. This solution raised significant concerns regarding privacy rights and the future of \textit{End-to-End Encryption (E2EE)}. 

This paper examines the technical feasibility, efficiency, and efficacy of the technological solutions also considering their deployment in an adversarial context. Furthermore, to estimate the impact on privacy rights, this discussion focuses on how this proposal relates to the current legislative framework by also reporting recent juridical cases to understand the interpretations given to privacy as a Fundamental Right.

Findings and subsequent discussion suggest that while child safety needs to be enhanced in digital contexts, the current way of implementing this safety could significantly harm privacy rights. 

Ultimately, this work calls for a more balanced approach to this matter and research on less intrusive technologies for the detection of criminal behaviors online.