\section{Introduction}

Since digital communication became common, the proliferation and spreading of \textit{Child Sexual Abuse Material (CSAM)} became a growing problem that required the intervention of both states and service providers. 

In response to this phenomenon, on 11 May 2022, the European Union Commission adopted a legislative proposal to fight the spreading of CSAM and online children's sexual solicitation (grooming)\cite{eu2023chatcontrol}. This proposed regulation is also known as Chatcontrol.

This legislation aims to detect these crimes by applying technologies based on ML classifiers for images and text on intercepted communications on targeted digital platforms. 

However, such a proposal raised debates and subsequent alarms on the potential effects that this proposal might have on the privacy of European citizens. In particular, concern was caused by the proposed scanning methodology that would endanger the use of \textit{End-to-End Encryption (E2EE)} \cite{effects}.

This work explores not only the legislative aspects of this proposal relating to privacy rights in the EU by studying the proportionality of the legislation, but also the technical feasibility, efficiency, and efficacy of the proposed infrastructure in solving the presented problem.

In Section \ref{s:related} this discussion firstly presents some relevant technical aspects relating to content-scanning technologies and algorithms. Successively, the topic has shifted to introduce parts of the European legislative framework concerning data and communication privacy. After that, the proposal object of this work is presented.

Moreover, In Section \ref{s:meth} the methodology used for gathering the data is discussed.

Section \ref{s:find} will show considerations on how the proposed system could raise an alarming number of false positives, how it would perform in an adversarial context, and how detection could be avoided.

Furthermore, an exploration of some judicial cases from the \textit{European Court of Human Rights (ECHR)} is presented to understand how privacy rights, national security, and other social matters are balanced.

Successively, in Section \ref{s:disc} a discussion on the legislation proportionality is given based on the results of the evaluation of the above-cited legislative framework, court rulings, and technical assessments on efficacy and efficiency, with also an opinion on a possible outcome for E2EE technology.

Finally, in Section \ref{s:concl} the conclusion is presented calling for further research in balancing child safety and privacy rights, and on less intrusive technologies for crime detection in digital communications.