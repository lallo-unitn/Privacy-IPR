\section{Conclusion}
\label{s:concl}

This work offered an overview of both the legislative and technical problems related to the European Commission CSAM Regulation Proposal, also known as Chatcontrol 2.0.

In particular, to understand the degree of interference that this proposal would cause to the privacy rights of EU citizens, both current related EU regulations and directives were taken into consideration. Furthermore, to interpret some fundamental articles, recent judiciary cases were reviewed.

Moreover, to assess the feasibility, efficacy, and efficiency of the proposed infrastructure from a technical standpoint, research data on classifiers, hashing algorithms, and content-scanning technologies were gathered. In particular, these data were critically compared with the ones reported in the impact assessment of the Commission and assessment of the Parliament.

While this work has limitations in terms of data analyzed and the expertise of the author in terms of legislative and judicial matters, it is my conclusion that this proposition presents an unbalanced solution to the problem of CSAM online spreading and sexual solicitation of children.

This is, not only because the control could easily be bypassed by those who had sufficient reason to learn how to do it, e.g. criminals, but also because the interference with the Fundamental Rights of the Union would not be proportional with the aim, the efficacy, and the efficiency of this proposal.

Ultimately, while researching this matter, not only did it become obvious how some consensus has to be reached on the balance between child safety and privacy, but also it became apparent that more research has to be carried out on non-intrusive (if any exists) or less intrusive technology for crime detection in digital communications.