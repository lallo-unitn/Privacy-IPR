\subsection*{European directive on legal protection of computer programs}

After the introduction of the Computer Software Copyright Act of 1980, also the European Economic Community (EEC) adopted a directive in order to guarantee a single market policy for software products. The Council Directive 31/250/EEC of 14 May 1991 on the legal protection of computer programs was roughly based on the aforementioned U.S.C. Copyright Act. We take into consideration the 2009 Directive that repealed the one from 1991. Following, the considerations that were made on the main articles.

\begin{itemize}
    \item \texttt{Article 1}: the computer programs are protected as literary works. The meaning of literary works is the same as the one defined in the Berne Convention. Discussion arose at the time of the making of this directive, as some experts thought that copyright was not a suitable tool to protect computer programs.
    \item \texttt{Article 2}: the exclusive rights are owned jointly by every natural person that took part in the creation of the computer program. Exception is made for master servant relations, in which the exclusive right is retained by the employers only. The case of the Free Software Foundation and the Fiduciary License Agreement was discussed.
    \item \texttt{Article 4}: the exclusive rights of the holder give them the power to authorize a set of actions defined in section 1(a), 1(b), and 1(c), in this way defining stronger exclusive rights than the one used in the US. It is important to note that Articles 5 and 6 define provisions for section 1(a) and 1(b).
    \item \texttt{Article 5}: in section 1, limitations to Article 4 are defined. Regarding the actions at Article 4, sections 1(a) and 1(b), no authorization to the lawful acquirer is required by the rightholder unless stated in contractual provisions. Note how the user agreement is a contract in the EU. Furthermore, section 2 defines the same limitations present in U.S.C Art. 17 \S 117.
    \item \texttt{Article 6}: defines how and when a program can be decompiled. This can be done only when the code needs to be studied in order to implement interoperability of an independently created computer program. Conditions are that the act must be performed by a licensed user or one in their behalf; the information to achieve interoperability has not been provided previously to the decompilation; the decompilation is limited to the parts fo the original program which are necessary to achieve the previously stated goal.
\end{itemize}

\subsection*{Free and Open Software}