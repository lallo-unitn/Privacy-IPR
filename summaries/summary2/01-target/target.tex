\AtBeginShipout{\AtBeginShipoutUpperLeft{%
  \put(\dimexpr\paperwidth-1cm\relax,-1.5cm){\makebox[0pt][r]{\framebox{Riccardo Gennaro} \framebox{Mat:47515} \framebox{SUMMARY 2}}}%
}}

\subsection*{European directive on legal protection of computer programs}

After the introduction of the \texttt{Computer Software Copyright Act} of 1980, also the European Economic Community (EEC) adopted a directive in order to guarantee a single market policy for software products. The \texttt{Council Directive 31/250/EEC} of 14 May 1991 on the legal protection of computer programs was roughly based on the aforementioned U.S.C. Copyright Act. We take into consideration the 2009 Directive that repealed the one from 1991. Following, the considerations that were made on the main articles.

\begin{itemize}
    \item \texttt{Article 1}: the computer programs are protected as literary works. The meaning of literary works is the same as the one defined in the Berne Convention. Discussion arose at the time of the making of this directive, as some experts thought that copyright was not a suitable tool to protect computer programs.
    \item \texttt{Article 2}: the exclusive rights are owned jointly by every natural person that took part in the creation of the computer program. Exception is made for master servant relations, in which the exclusive right is retained by the employers only. The case of the Free Software Foundation and the Fiduciary License Agreement was discussed.
    \item \texttt{Article 4}: the exclusive rights of the holder give them the power to authorize a set of actions defined in section 1(a), 1(b), and 1(c), in this way defining stronger exclusive rights than the one used in the US. It is important to note that Articles 5 and 6 define provisions for section 1(a) and 1(b).
    \item \texttt{Article 5}: in section 1, limitations to Article 4 are defined. Regarding the actions at Article 4, sections 1(a) and 1(b), no authorization to the lawful acquirer is required by the rightholder unless stated in contractual provisions. Note how the user agreement is a contract in the EU. Furthermore, section 2 defines the same limitations present in U.S.C Art. 17 \S 117.
    \item \texttt{Article 6}: defines how and when a program can be decompiled. This can be done only when the code needs to be studied in order to implement interoperability of an independently created computer program. Conditions are that the act must be performed by a licensed user or one in their behalf; the information to achieve interoperability has not been provided previously to the decompilation; the decompilation is limited to the parts fo the original program which are necessary to achieve the previously stated goal.
\end{itemize}

\subsection*{Free and Open Software}

The definition of free software given by Richard Stallman in a 2007 interview is articulated in four points

\begin{itemize}
    \item "freedom to run the program as you wish": no exclusive rights on the usage of the program;
    \item "freedom to study the source code and then change it so that the program does what you wish": no exclusive rights on accessing the source code;
    \item "freedom to distribute copies to others": users can make copies and distribute them freely;
    \item "the freedom to distribute copies of your modified versions"; users can modify and distributes copies of that modification.
\end{itemize}

This principles were firstly implemented in the \texttt{General Public License Version 1} (GPLv1) in 1989. In this version, the implementation of the fourth principle defined a derivative work as a piece of code containing a GPLed work or using a GPLed library. This became a problem when the GNU C library was to be released. In fact, since the GPL is a copyleft license, every project that used the GNU C library has a linked library was required to ship under the GPL. This meant that no copyrighted C-written software that used the GNU C library could be made.
Strategically, to promote the the usage of this library, in 1991 the FSF decided to release the \texttt{Lesser GPL} (LGPL), a copyleft license that redefined the concept of derivative work. Under the LGPL a piece of code that used a linked GPLed library was not to be considered a derivative work anymore.

The same year of the LGPL release, the first largely adopted free kernel was released. In fact, 1991 was the year in which Linus Torvalds released the first version of the Linux monolithic kernel with no license. Only in 1992, with version 0.12 the Linux kernel shipped under the GPL.

But the Linux kernel is not the only free kernel that has been released. By aggregating more then ten years of work, starting from the TCP/IP implementation for UNIX OS commissioned by DARPA in 1979, the \texttt{Berkeley Software Distribution} (BSD) released in 1992 the BSD kernel. Unfortunately, since BSD entered in a legal battle with AT\&T for trademark and copyright infringement over the usage of the name UNIX OS, the development of this kernel fell behind the Linux one. This caused BSD to never enter mainstream usage. A peculiarity of the BSD OSs is that this family of operating systems ship with the BSD license, which is non-copyleft.

Nowadays, more than 60 free-software licenses exist and are approved by the Open Source Initiative (OSI). The problem of conflicting licenses, i.e. copyleft vs. non-copyleft, persists.

\subsection*{File sharing}

File sharing is formally defined as the activity of giving unauthorized access to files containing copyrighted work. The act of sharing and the act of accessing such material without the consent of the rightholder is considered copyright infringement and is punishable by law. There exist two types of copyright infringement: \texttt{direct} and \texttt{secondary}. Primary infringement involves a direct infringement by the defendant, while secondary infringement happens if someone facilitates another person or group in infringing on a copyright.

After a digression on "analog" file sharing, examples of compression algorithm were given in order to explain how file sharing became possible in the late 90's using low-bitrate DSL technology. Three study cases were analyzed: Napster, Pirate Bay, and Megaupload.

\subsubsection*{Napster}

Napster (1999-2002) was a peer-to-peer file sharing application. The shared material was mostly copyrighted music encoded in compressed mp3 format. The mainstream use of this platform allegedly damaged a number of well-known artists, while also giving visibility to the emerging band Radiohead. Consequences aside, Napster was sued for secondary copyright infringement. Differently from fully peer-to-peer protocols like Gnutella or BitTorrent, Napster used a centralized directory to index peers with chunks of the searched files, making them responsible of policing the shared content. Napster closed in 2002 after losing the lawsuit. 

\subsection*{Pirate Bay}

Pirate Bay is a web directory for BitTorrent magnets and .torrent metadata files. Differently from Napster, BitTorrent do not provide a service for indexing peers or data chunks. To access a shared file, one must download its metadata from a web page like Pirate Bay where third parties upload magnets linked to the file they're sharing. Doing so, Pirate Bay is liable for secondary infringement and it already has been under trial. Nonetheless, like multiple other alternatives, it continue to exist by constantly changing its domain.

\subsubsection*{Megaupload}

Megaupload was a centralized web service for file hosting and sharing. Megaupload offered a freemium subscription plan: the free subscription offered a low-bitrate in order to push the user to subscribe to the premium plan. Its functionalities were comparable to the ones of Dropobox and Google Drive. Differently from these legitimate services, Megaupload was lax about DMCA enforcement and for this reason was eventually shut down.