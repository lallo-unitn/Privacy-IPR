\subsection*{Law's general history & structure}

The law can be defined in two ways: as "a body of rules to regulate the social behavior of a community of people in a given territory", or as "the formal regime that orders human activities and relations through the systematic application of the force of a governing body and the society it rules over". From the two definitions we can derive that law is a set of rules, it should regulate the social behavior of a given community in a given territory. Furthermore, by the second definition, the State has a monopoly over the use of violence, which is used as a remedy when a rule is broken.

Two fundamental events that shaped the Western Law are the French and the American revolutions. With these regime changes, democracy was chosen as a political system. Now the People have the power to write and enforce laws. For this reason, the Division of Powers became necessary, dividing these powers into Legislative, Judiciary, and Executive.

During and following this period, a large set of rules were made into laws. As a consequence, the phenomenon of codification took place. Codification stands for the domain-based logical separation of rules in different codes. Western Law can be divided into public and private. More specifically, Public Law is dived in

\begin{itemize}
    \item Constitutional Law: states explicitly how the state is organized and provides an explicit set of individual rights.
    \item Criminal Law: called Penal Law in Europe, is the body of rules that punishes criminals for committing offenses against the state or other individuals or organizations.
    \item Administrative Law: it rules over the functions and powers of the State agencies. Together with constitutional law defines the organizations and the power of the state over private citizens.
\end{itemize}

Also, Private Law is divided into Civil, Family, and Commercial Law. In particular, Civil Law regulates economic relationships among persons and private organizations. Civil Law includes Property and Contract Law, and for this reason, it is an object of our discussion since Intellectual Property Rights (IPRs) are an instance of Property Law and - arguably, in Europe - of Contract Law.

Intellectual Property and Property can be said to be similar as they define exclusive rights over given resources that can be traded through the use of a contract called \texttt{contract of sale} for material resources, and \texttt{license} for intellectual properties.

To trade between entities belonging to different states, an international regulation is needed. IPRs were the object of one of the first international regulations. Also, a particular instance of International Law is the European Law; this Law deals with statutory law created within the
European Union and is divided into Directives, Regulations, and Decisions.

\subsection*{History of Copyright Law}

Copyright Law is based on the objective of promoting learning and the progress of science. This mission can be traced back to the \texttt{Statute of Anne(1710)} and the \texttt{U.S. Constitution art. I, Sec. 8, cl. 8}. So, the creation of copyright in terms of an exclusive right given to authors is not meant to reward authors but is meant to promote the progress of science and useful arts. But, at the same time, this law provides an author with an exclusive right. This conflict was resolved by limiting the right to exclude.

But, to understand the origin of copyright we need to go as far back as the invention of printing in the 15th Century. This technological revolution put in the hands of stationers the power to spread information and culture. At the same time, an (not only) ideological war was taking place between Roman Catholics and Anglican Christians. For this reason, Queen Mary granted the Charter of the Stationers' Company to control the spreading of ideas via the normative power bestowed upon the stationers.

In part, the stationers took advantage of their monopoly by creating "their copyright, shaped it to their ends, and kept control of it for themselves" (Ray Patterson). For instance, they created an internal rule according to which, if you wanted to print a book, you needed to register the title of that book with the stationers' company, and that registration gave the exclusive right of the printing of the given title forever.

Only with the rise of the Enlightenment a critique of this regime was brought up (see Locke's critique on the stationers). For this reason, the Statute of Annete was made into law to guarantee the spreading of knowledge and culture, and to protect the author's work from the stationer's monopoly.

After this statute, numerous regulations were made to expand and globalize copyright laws. The \texttt{Berne Convention(1886)} was an important milestone for this process of globalization and agreeing on a common set of legal principles regarding copyright. It is important to note that the U.S.A. did not take part in the Convention. This is because the States, at the time, were not the cultural superpower we came to know during the 20th Century. For this reason, the Americans decided not to follow the international regulations to import intellectual works without having to abide by such laws. This changed only when they started to export ideas and culture.

Finally, in 1893 the Bureau for the Protection of Intellectual Property was founded, and later in 1967, it was renamed as \texttt{World Intellectual Property Organization (WIPO)}, which produced in 1996 the WIPO Copyright Treaty.


\subsection*{Copyright Law}

So, we must define the object to protect. Copyright Law(CL) protects the expression of an idea and not the idea itself. For protecting ideas, patents are used. This is because authors are supposed to find new ideas or to better existing ones. So, applying CL to ideas would be counterproductive. This constitutes one of the limitations of the right to exclude.

To comply with the objective that was stated at the beginning of this document, limitations must be applied to the exclusive rights of copyright. Following we find a taxonomy of these limitations

\begin{itemize}
    \item \texttt{Expression/Idea Dichotomy}. As explained before, it limits the protection to the expression of a given idea (cfr. WIPO Copyright Treaty Art 2 and § 102 (b) Title 17 U.S. Code).
    \item \texttt{Fair Use}. The right to exclude is limited depending on the usage of the protected material. There is no single definition of fair use. In the States, one can freely use copyrighted works for criticism, commenting, teaching - et similia - purposes. Instead, in Italy there is little room for interpretation as the Law specifically lists exceptions and not a general clause (see limit of fair use of books).
    \item \texttt{First Sale Doctrine}. This limitation regulates the first and the subsequent sales of protected objects. In particular, the first sale must take place between entities authorized by the copyright holder; whilst, the subsequent sales can take place between unauthorized entities and other third parties (second-hand market).
    \item \texttt{Originality}. In the case of Feist Publications, Inc. v. Rural Telephone Service Co., 499 U.S. 340 (1991), the Supreme Court ruled that copyright law only protects original, creative expressions, not mere facts or data. Similarly, italian Art. 1 Legge 633/1941 establishes that "Sono protette ai sensi di questa legge le opere dell'ingegno di \texttt{carattere creativo} [...], qualunque ne sia il modo o la forma di espressione."
\end{itemize}

\subsection*{Copyright Law and Software}

While the history of the first computers, UNIX, MINIX, and FreeBSD was discussed during the last lecture, this part wants to summarize the legislation about copyright related to computer programs by dividing it as pre and post \texttt{Copyright Act(U.S. Congress, 1976)}.

Before the 1976 act, copyright was regulated by the 1909 Copyright Act, further circular \texttt{Copyright Office Circular No. 61 (1964)} and previous court case \texttt{White-Smith Music Publishing Company v. Apollo Company, 209 U.S. 1 (1908)} - the piano rolls company case. These documents roughly stated that a protected intellectual work must be "intelligible by human beings", and such that there is the possibility to "see and read with the naked eye".

The problem with this piece of legislation is that every object code produced by a compiler cannot be protected by copyright under this definition of the law. Only source code can be since it is intelligible. By 1976 computer science was already a big industry, and most of the time, products were sold as object code to conceal the source code. However, the object code wasn't protected as per the previously stated reasons.

It follows that the 1976 Act was introduced to make it possible to protect also object code by changing the definition of literary work. The new definition stated that the work could be "expressed in words, numbers, or other verbal or numerical symbols or indicia".

Successively, further specifications to limit the exclusive rights were made under the \texttt{Computer Software Copyright Act of 1980} produced by CONTU, which included as fair use the creation of a backup copy, and the possibility to adapt the software for personal needs.

