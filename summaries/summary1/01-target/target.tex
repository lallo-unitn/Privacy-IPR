\subsection*{Law's general history & structure}

The law can be defined in two ways: as "a body of rules to regulate the social behavior of a community of people in a given territory", or as "the formal regime that orders human activities and relations through the systematic application of the force of a governing body and the society it rules over". From the two definitions we can derive that law is a set of rules, it should regulate the social behavior of a given community in a given territory. Furthermore, by the second definition, the State has a monopoly over the use of violence, which is used as a remedy when a rule is broken.

Two fundamental events that shaped the Western Law are the French and the American revolutions. With these regime changes, democracy was chosen as a political system. Now the People have the power to write and enforce laws. For this reason, the Division of Powers became necessary, dividing these powers into Legislative, Judiciary, and Executive.

During and following this period, a large set of rules were made into laws. As a consequence, the phenomenon of codification took place. Codification stands for the domain-based logical separation of rules in different codes. Western Law can be divided into public and private. More specifically, Public Law is dived in

\begin{itemize}
    \item Constitutional Law: states explicitly how the state is organized and provides an explicit set of individual rights.
    \item Criminal Law: called Penal Law in Europe, is the body of rules that punishes criminals for committing offenses against the state or other individuals or organizations.
    \item Administrative Law: it rules over the functions and powers of the State agencies. Together with constitutional law defines the organizations and the power of the state over private citizens.
\end{itemize}

Also, Private Law is divided into Civil, Family, and Commercial Law. In particular, Civil Law regulates economic relationships among persons and private organizations. Civil Law includes Property and Contract Law, and for this reason, it is an object of our discussion since Intellectual Property Rights (IPRs) are an instance of Property Law and - arguably, in Europe - of Contract Law.

Intellectual Property and Property can be said to be similar as they define exclusive rights over given resources that can be traded through the use of a contract called \texttt{contract of sale} for material resources, and \texttt{license} for intellectual properties.

To trade between entities belonging to different states, an international regulation is needed. IPRs were the object of one of the first international regulations. Also, a particular instance of International Law is the European Law; this Law deals with statutory law created within the
European Union and is divided into Directives, Regulations, and Decisions.

\subsection*{History of Copyright Law}

Copyright Law is based on the objective of promoting learning and the progress of science. This mission can be traced back to the \texttt{Statute of Anne(1710)} and the \texttt{U.S. Constitution art. I, Sec. 8, cl. 8}. So, the creation of copyright in terms of an exclusive right given to authors is not meant to reward authors but is meant to promote the progress of science and useful arts. But, at the same time, this law provides an author with an exclusive right. This conflict was resolved by limiting the right to exclude.

But, to understand the origin of copyright we need to go as far back as the invention of printing in the 15th Century. This technological revolution put in the hands of stationers the power to spread information and culture. At the same time, an (not only) ideological war was taking place between Roman Catholics and Anglican Christians. For this reason, Queen Mary granted the Charter of the Stationers' Company to control the spreading of ideas via the normative power bestowed upon the stationers.

In part, the stationers took advantage of their monopoly by creating "their copyright, shaped it to their ends, and kept control of it for themselves" (Ray Patterson). For instance, they created an internal rule according to which, if you wanted to print a book, you needed to register the title of that book with the stationers' company, and that registration gave the exclusive right of the printing of the given title forever.

Only with the rise of the Enlightenment a critique of this regime was brought up (see Locke's critique on the stationers). For this reason, the Statute of Annete was made into law to guarantee the spreading of knowledge and culture, and to protect the author's work from the stationer's monopoly.



